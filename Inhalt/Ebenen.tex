
\chapter{Fundamental Concepts}
\label{cha:fundamentals}
\textbf{TODO: I -> We, no need to always say will, sometimes use passive to switch it up}
In this chapter we lay out the fundamental concepts, which are necessary in order to understand the subsequent chapters.
First we explain Stream Processing in \ref{sec:stream-processing}, afterwards the concept of the self-adaptive systems will be presented and explained in section \ref{sec:self-adaptive} 
followed by a brief explanation of the MAPE-K loop in \ref{sub:mape} to conclude this chapter.

    \section{Stream Processing}
    \label{sec:stream-processing}
    In this section we will split the concept of stream processing into three further components.
    In \ref{sub:sps} we will then define stream processing systems, explain how they work and give examplary fields of application.
    Afterwards in \ref{sub:dsms} we will move onto the topic of data stream management systems and finally in \ref{sub:requirements} we talk about
    some requirements that SPSs should meet.
    
        \subsection{Data Stream Management Systems}
        \label{sub:dsms}
        \ifbool{final}{}{\textbf{\color{green}THIS IS FINAL}
        
        }
        % - Streams not saved to disk (Saving to disk and retrieving has high latency), instead kept in memory (low latency)
        % - Some data might still be archived, however it does not impact performance, as it is done parallel to the streaming processing
        % - Sometimes access to archived data in DBMS needed, however this can be cached or efficiently accessed

        \begin{figure}[h]
            \label{fig:dbms_dsms}
            \centering
            \includegraphics[width=1.0\textwidth]{Bilder/dbms_dsms.png}
            \caption{
                   left: architecture of a DBMS; right: architecture of a DSMS
                   }
        \end{figure}

        A data stream management system is a tool designed to manage continuous data streams.
        DSMSs are comparable to DBMS, however, there are differences one must take note of.

        \quad Most notably the data being fed into a DSMS varies greatly from the data being inserted into a DBMS; while a DBMS receives a finite predetermined amount 
        of data, a DSMS can theoretically receive an infinite continuous stream.
        \\
        Secondly, DSMSs operate at a much greater speed than traditional DBMSs, because the data is processed immediately, while in the DBMS approach
        data is first saved to a disk and then queried before further processing can occur, leading to two I/O actions. This behavior, coupled with an incoming 
        high velocity stream, might lead to lags or even failure to yield accurate results\cite{StreamBookQuality}. 
        However, a DSMS might want to archive certain elements from the stream. For this, there are two possibilities;
        Either data is being cached in memory, yielding much faster query results, or it is archived in additional databases parallel to the streaming process.
        Thus the DSMS architecture is superior to the DBMS architecture when measuring latency.
        When data is stored, often only synopses are stored, in an attempt to summarize the data and utilize that knowledge in form of a \gls{state} for the queries.
        The difference in these architectures is also shown in figure \ref{fig:dbms_dsms}\cite{StreamBookQuality}.
        \\
        Thirdly, similar to a DBMS a DSMS can also be queried using a streaming query language, however, queries installed on a DSMS are continuous 
        and will be executed as long as it is uninstalled and there are restrictions on which operations can be executed.
        Since the system can not assume a finite amount of data, it processes each data item as it arrives, so operators that require the entirety of a data 
        set, such as the \textit{join, aggregation} operators, prove problematic when it comes to streams, 
        as they can only return a result once the stream has ended\cite[p.12]{StreamBookQuality}.
        These queries are refered to as \textit{blocking} queries. In order to use them, one must convert them into non-blocking queries.
        One solution for this has been introduced in the form of windows, which are explained in \ref{sub:sps} \nameref{sub:sps}.

        \quad There are a few more functional differences, which Panigati, Schreiber and Zaniolo highlight, as shown in table 2.1.

        \begin{table}[h]
            \centering
            \label{tab:dbms-dsms}
            \begin{tabular}{|c|c|c|} \hline
                \textbf{Feature} & \textbf{DBMS} & \textbf{DSMS} \\ \hline
                Model & Persistent data & Transient Data \\ \hline
                Table & Set or bag of tuples & Infinite sequence of tuples \\ \hline
                Updates & All & Append only \\ \hline
                Queries & Transient & Persistent \\ \hline
                Query answers & Exact & Often approximate \\ \hline
                Query evaluation & Blocking and non-blocking & Non-blocking \\ \hline
                Operators & Fixed & Adaptive \\ \hline
                Data processing & Synchronous & Asynchronous \\ \hline
                Concurrency overhead  & High & Low \\ \hline
            \end{tabular}
            \captionbelow{Functional comparison of DBMS and DSMS\cite{Panigati2015}} 
        \end{table}

        \subsection{Stream Processing Systems}
        \label{sub:sps}
        \ifbool{final}{}{\textbf{\color{red}THIS IS 90\% FINAL}
        
        }
        % TODO Erklärung von Windows irgendwie einbringen (Hier noch überlegen wo am besten)
        % TODO Challenges, die SPSs facen

        % Topical sentence
        In this section we will define stream processing systems, expound what data looks like in SPS. We also explain what SPSs
        are capable of and give an example of a SPS. Furthermore windows will be explained and the concept of adaptation in SPSs will be laid out.
        In the end of the section we will present some of the challenges that SPSs face.

        % Todo: Batch processing kurz erläutern, higher latency b/c larger amounts of data are processed at once??
        \quad Stream processing systems are systems designed to handle large amounts of data in (near) real-time fashion with low latency.
        SPSs were designed in order to fulfil the needs for fast processing, which was not solved by the previous batch processing systems.
        Batch processing systems were able to process huge amounts of data as well, however the difference that needs to be emphasized is the latency.
        Stream processing systems complete their tasks faster, by processing each element as it arrives as opposed to the batch system, which collects elements 
        and then processes the data in batches.
        The data is fed to the system in form of streams, which are defined as an potentially infinite, countable amount of \textit{data tuples}. 
        The streams usually originate from a \texit{data source} which, for example, can be a type of sensor, e.g. a simple temperature sensor or more complex sensors, 
        such as an EEG-sensor, market orders or network traffic.
        \\
        A data tuple is an object whose internal structure is defined by a \texit{data scheme}. 
        Objects of a data stream share a common structure, also called \textit{stream schema}.
        Tuples consist of a set of typed attributes and their respective values. One can imagine a tuple as a java object without any operations.\cite{fundamentals}
        Each individual data tuple is processed by one or more \textit{operators}.
        An operator is fed tuples from incoming streams through its input port, performs an operation and generates an output stream through its output port, 
        consisting of the processed tuples. Operators are able to perform various different operations, 
        as shown in \cite[p.49]{fundamentals}, some of which include: aggregation, splitting and merging streams, 
        logical and mathematical operations and custom data manipulations.
        \\
        The results of a streaming application is then fed into one or multiple \textit{data sinks}, which could for example be a data visualization engine 
        or an automated stock trading engine.
        
        \quad In order for a SPS to perform as fast as possible, it should have the ability to adapt to situations.
        There are many strategies of adaptation, which can be broken down into three subcategories; \textit{processing-}, \textit{data-} and 
        \textit{resource adaptation} \cite[p.8 f.]{QIN20191}. 
        An examplary resource adaptation would be a \textit{processing scaling} adaptation, as shown in figure \ref{fig:sps_parallel_normal}, 
        where, if computational resources allow it, an operator may be replicated in order to process data faster by
        making use of the principle of parallelity. Technically, there should be a splitting operator placed before, 
        as well as a merging operator after the replicated note. These have been left out for the sake of brevity.
        \\
        An example for data adaptation would be \textit{load shedding}, a strategy where some data tuples are dropped in order to free up computational resources.
        However, one must take into consideration that one possibly sacrifices accuracy for the sake of performance, as more data generally leads to more accurate results.
        Furthermore, one must take note of the resource cost to calculate which tuples to drop in order to have the lowest impact on the quality of service.
        % TODO Maybe an example here for load shedding?
        \\
        For resource adaptation we will look at \textit{dynamic resource allocation}. Dynamic resource allocation is a stratey, in which the available computational 
        resources are constantly reassigned, so as to allocate the memory in the most efficient way. For example a node facing a high data load at the moment might 
        be allocated more RAM, GPU or CPU power in order to process its load faster. For this to work some resources have to be freed first, e.g. by deducting some
        computational power from a node facing less input at the time. With this strategy, one must take into consideration the resources and time spent calculating 
        the optimal allocation of resources.

        % Challenges
        \quad By virtue of the fields in which stream processing is applied, SPSs face a series of challenges.
        
        \quad As SPSs are used as a means to process big data in a (near) real-time fashion, we will take a system that predicts changes in stock values 
        of companies based on the public's opinion, which is extracted from tweets, as an examplary application.
        \\
        The application receives new \gls{tweet}s as input, then performs operations such as clustering and filtering for certain terms or company names. 
        Afterwards a sentiment analysis is performed on said tweets and a prediction for stock prices is computed and then sent 
        to an automatic stock trading application and stock traders for manual reviews.


        \begin{figure}[h]
            \label{fig:sps_parallel_normal}
            \centering
            \includegraphics[width=1.0\textwidth]{Bilder/sps_parallel_normal.png}
            \caption{
                    Left: An example for an SPS displayed as a directed acyclic graph;
                    Right: Same SPS with introduced parallelity in one operator, marked gray for visibility;
                    Circles on the left depict data sources, circled on the right depict data sinks, squares are operators/processors, and arrows are streams.
                    }
        \end{figure}

        % \begin{figure}
        % \label{fig:stream-processing-system}
        % \centering
        % \includegraphics[width=1.0\textwidth]{Bilder/stream-processing-system.png}
        % \caption{
        %         Overview of a basic Stream Processing System
        %         }
        % \end{figure}  

        % Subsection Requirements for Stream Processing Systems
        % Should quickly go over the requirements and explain why they matter to us
        \subsection{Requirements for Stream Processing Systems}
        \label{sub:requirements}
        \ifbool{final}{}{\textbf{\color{red}THIS IS NOT FINAL}
        
        }

        Notes: What are the requirements, why do they matter to us (Elaborate on this)

        Due to the nature of the fields in which SPS are used, there are important requirements that SPS should meet in order to be viable, 
        which Stonebraker et al. point out in \cite{Stonebraker:2005:RRS:1107499.1107504}, of which the ones most important to us can be summarized as the following:
        
        % Enumeration of requirements for SPS + explanations why they matter to us
        \begin{enumerate}
        \label{enum:requirements}
            \item \textbf{Keep the Data Moving:} 
                In order to minimize latency, data must not be stored, as these are costly operations.
            \item \textbf{Handle Stream Imperfections:} 
                Expecting only perfect data is utopian, so one must prepare the system with built-in mechanisms for data that might be missing or out-of-order.
            \item \textbf{Integrate Stored and Streaming Data:} 
                For an SPS to be able to perform comparisons between "predecessor" data and current data, operators must keep an efficiently manageable state.
            \item \textbf{Guarantee Data Safety and Availability:} 
                Recovering from a failure is detrimental for real-time data processing, so a system must be in place to guarantee the highest availability possible.
            \item \textbf{Process and Respond Instantaneously:} 
                Systems must be highly optimized in order to provide (near) real-time responses.
            \item \textbf{Partition and Scale Applications Automatically:} 
                Systems must be able to be split across multiple machines and threads.
                The system must also be able to automatically scale and distribute the load across the machines.

        \end{enumerate}

    \section{Self-Adaptive Systems}
    \label{sec:self-adaptive}
    \ifbool{final}{}{\textbf{\color{green}THIS IS FINAL}
    
    }{}
    
    % \textbf{TODO: Needs more; e.g why self-adaptive(already mentioned in intro), Software engineering perspective on self adaptivity, challenges}
    % Definition of self-adaptive systems
    % Architecture often based on mape in different patterns
    % applied in xx industries
    Cheng et al. define self-adaptive systems as
    \begin{quotation}
        ``[\ldots] systems that are able to adjust their behaviour in response to their perception of the environment and the
        system itself [\ldots]``\cite[p.1]{Cheng:2009:SES:1573856.1573858}.
    \end{quotation}
    
    \quad Self-adaptive systems are oftentimes based on the \nameref{sub:mape} [p.\pageref{sub:mape}] pattern.
    Adaptive Systems have a wide variety of possible application areas: adaptable user interfaces, autonomic computing, multi-agent systems \cite{Cheng:2009:SES:1573856.1573858}, 
    biologically inspired computing, robotics \cite{10.1007/978-3-319-59480-4_44}, streaming applications and a lot more.

    \quad An examplary application would be a scenario, in which population and food capacities are given and evolving over time, due to births, deaths, changes in demographics 
    and changes in weather and harvest respectively. A system would have to adapt to these changes in its environment in order to ration the food properly.

    \quad Self-adaptivity however presents itself as challenging when looked at from a software engineering perspective.
    Cheng et al. illustrates some of these challenges in \cite{Cheng:2009:SES:1573856.1573858}, which we will discuss in the following paragraphs for both the 
    requirements engineering and the architecture disciplines.
    \\
    As the system adapts, it changes, and with it do the requirements. This leads to the fact, that language changes in requirements engineering.
    Usually requirements are formulated in a strict manner `\textit{The system shall\ldots}`. This has to be relaxed and changed, Cheng et al. propose the following 
    option amongst others: `\textit{The system may do this\ldots or it may do that\ldots}`.
    \\
    More importantly, requirements engineering must define exactly what is monitored as well as high-level goals, which need to be fulfilled at any given environmental condition.
    In addition, one must also specify under which circumstance what kind of adaptation has to be performed and how said adaptation is realized.
    Due to the increased amount of requirements and the everchanging nature of an adaptive system, it gets harder to trace whether requirements are fulfilled, adding to the complexity.
    \\
    One must also take into account that it is nearly impossible to forecast every situation that a system may face, therefore challenging requirements engineers with incomplete information.

    \quad The complexity is also increased for engineers and architects, as there are new factors to consider as well.
    Single MAPE-K loops do not scale efficiently for large systems, one must find a solution, often times multiple control loops are constructed.
    Good practice however suggests usiing just a single control loop\cite{accidents}, so one should try to minimize the amount of control loops used and
    make them independent from one another.
    
    
    \quad Architects are faced with a number of new decisions;\\
    Is a centralized or decentralized architecture better to achieve one's goals? Are the control loops independent from one another?
    Is a hierarchical approach appropriate? Good practice suggests using just a single control loop\cite{accidents}, so can the amount of control loops be scaled down?
    \\
    Research is needed to battle some of the issues that self-adaptive systems and control structures face, such as structural arrangements. 
    It is therefore necessary to explore more architectures, an example will be shown in \ref{sec:hierarchical}, in which a decentralized hierarchical approach was
    chosen to combat said issues. In addition, evolving an existing system into a self-adaptive system currently requires a lot of work, a possible area of research 
    would be finding a way to 'inject' self-adaptivity into pre-existing systems.
    
    \quad Cheng et al propose that control loops be made explicit and kept seperate so as to 'seperate the concerns of functionality from the concerns of self-adaptation'\cite[p. 16]{Cheng:2009:SES:1573856.1573858}.
    Furthermore, they emphasize the importance of understanding control loops as the most important tool in self-adaptive systems in order to mature the research area of self-adaptive systems.
    
    \subsection{MAPE-K Loop}
    \label{sub:mape}
    \ifbool{final}{}{\textbf{\color{green}THIS IS FINAL)}
    
    }{}
    % Explain the MAPE-K Loop as it is a valuable basis/reference architecture for many different approaches in adaptive systems
    % Rough explanation of what it is, where it is used
    % explain different "stages" (m, a, p, e)
    % Explain -K extension
    The MAPE-K Loop was introduced by IBM \cite{Kephart:2003:VAC:642194.642200} and refers to a proposed solution for self-adaptive or autonomic systems.
    This model has since become the basis or reference architectural pattern for many self adaptive systems, which I will show in the third chapter.
    The acronym MAPE-K refers to the components that make up the model:
     \begin{figure}[hbt]
        \label{fig:mape}
        \centering
        \includegraphics[width=0.45\textwidth]{Bilder/mape.png}
        \caption{
                Overview of the MAPE-K-Loop\cite{Kephart:2003:VAC:642194.642200}
                }
    \end{figure}  
    \begin{enumerate}
    \label{enum:mape}
        \item \textbf{M}onitor: 
            The \textit{Monitor} component gathers data about the system and its environment, aggregates and filters it.
        \item \textbf{A}nalyze: 
            The \textit{Analyze} component analyzes the previously gathered data and determines whether or not an adaptation should be performed.
            The decision is made based on performance or cost gain and should include the adaptation cost as well.
            This component's analysis is influenced by the \textit{Knowledge} base.
        \item \textbf{P}lan: 
            If the choice to adapt the system has been made, the \textit{Plan} component then decides how to reconfigure the system.
            Once the decision has been made, the information is then forwarded to the \textit{Execute} component.
        \item \textbf{E}xecute: 
            Given the \textit{Plan} component's decision, the \textit{Execute} component then executes said plan and the loop 
            returns to the initial monitoring state.
        \item \textbf{K}nowledge: 
            Represents the knowledge base, which is shared between the other components.
            This base is created by the \textit{Monitor} component and contains information in the form of metrics, policies, symptoms and logs.
    \end{enumerate}
   


\chapter{Approaches for Self-Adaptive Architectures in Stream Processing}
\label{cha:approaches}
\textbf{TODO: How does it work? FOCUS ON ARCHITECTURE IN THIS CHAPTER!!}
Explain that this chapter showcases a few select strategies, which are then elaborated on further in the subchapters
Question: Even more approaches? e.g. Master-Slave pattern or Coordinated Control pattern (Both MAPE based)?
\textbf{Add and explain a few more MAPE Based architectures}

    \section{Dhalion}
    \label{sec:dhalion}
    In this section we will explain Dhalion \cite{dhalion}
    Quick Introduction to Dhalion, this chapter will deal with the Dhalion paper.
    NOTE: Explain what Dhalion is, where its used

        \subsection{An Outline of Heron}
        \label{sub:heron-outline}
        Small outline of Heron, as Dhalion is built on top of Twitter's Heron.

        \subsection{Dhalion's Architecture}
        \label{sec:dhalion-architecture}
        \textbf{TODO: Diagram, explain it here}
        Explanation of Dhalion's Architecture \textbf{KERNPUNKT DER SECTION DHALION}

        \subsection{Discussion of Dhalion}
        \label{sec:dhalion-discussion}
        Discuss the approach and compare it to the reference architecture (Mape?)
        \textbf{TODO: Maybe discuss how they evaluate, look at metrics relevant to architecture}

    \section{Decentralized Self-Adaptation}
    \label{sec:hierarchical}
    \textbf{TODO: Incorporate same structure as above}
    In this section we will explore a decentralized hierarchical architecture as described by Cardellini et al in \cite{cardellini}.
    \\
    In \ref{sec:edf} we will briefly define the terms 'decentralized' and 'hierarchical', followed by a description of the proposed architecture.
    \\
    Afterwards we will discuss the architecture, talk about the advantages and possible complications. Further we will compare it to its reference architecture in \ref{sec:discussion-edf}.


        \subsection{Elastic and Distributed DSP Framework}
        \label{sec:edf}
        \textbf{TODO: Possibly change title of this, add diagram, explain thoroughly (decentralized etc.), extract their design patterns}
        % Explanation decentralized
        % Explanation hierarchical
        % Present architecture + diagram
        % -> Single MAPE loop is bottleneck and bad for scalability [p. 4]&[cheng et al]
        %    -> As per wayns et al [ref. 39] (patterns for multiple loops??)
        % -> Organized according to hierarchical pattern for decentralized control
        %   ->  Higher-level mape components control subordinate MAPE components (? Noch nicht so ganz klar)
        % -> at lower level & faster time-scale: operator - and node manager
        % -> Operator manager
        %   -> in charge of controlling adaptation of a single operator using local MAPE loop
        %   -> Operator Monitor (M) -> Local Reconfiguration Manager (A+P) -> Reconfiguration Actuator (E)
        %   -> possible adaptations: Scale-in scale-out
        % -> Node manager
        %   -> per node
        %   -> oversees working conditions of a computing resource
        %   -> Task is to avoid over-utilizitation of computing resource
        %       -> Migrates tasks/hosted operator repliclas to neighbors if necessary
        % -> Both Managers issue a request for adaptation to higher layer
        % -> Higher level slower time scale: Application Manager
        % -> Centralized entity that coordinates adaptation of the whole application through global mape loop
        %   -> Application Monitor (M) -> GLobal Reconfiguration Manager (A+P) -> Global actuator (E)
        % -> pause-and-resume approach (erklaeren) [40]
        % -> stateful migration protocol (erklaeren) [17]
        %   -> Downtime 
        % -> General and no internal policies yet defined


        Explanation of the EDF, their architecture for elastic DSP apps \textbf{KERNPUNKT DER SECTION}

        \subsection{Discussion of EDF}
        \label{sec:discussion-edf}
        % Talk about their approach to finding the architecture for geo-distributed
        % -> Master-slave pattern
        %   -> Centralized Analyze & Plan component, rest decentralized and multiple instances
        %   -> Easier design & reconfiguration [ref. 17]
        %   -> However, centralized component is still a bottleneck, may introduce severe communication overhead when communicating with decentralized, geo-dsitributed components
        % -> Coordinated control pattern
        %   ->  As previously explained even a single centralized component can be non-feasible
        %   ->  multiple decentralized mape loops, with each loop overseeing a certain part of system
        %   ->  loops might need to interact/coordinate with one another to reach joint decisions
        %   ->  Failure to communicate may lead to too frequent adaptation decisions
        %   ->  conversely Communication too much reduces performance
        %   ->  harder to design control policies due to decentralization
        % -> Diagram to showcase master-slave and control pattern (MAPE loops)
        % Talk about advantages of the architecture
        % possible complications
        % compare it to the reference architecture
        % refer to cheng et al in self-adaptive systems chapter
        % 

    \section{Some Other Architecture}
    \label{sec:soa}
    NOTE: In this chapter I will explain another architecture/approach to self-adaption, yet to be researched


    \section{Some Other Architecture2}
    \label{sec:soa2}
    NOTE: In this chapter I will explain another architecture/approach to self-adaption, yet to be researched


    \section{Title??}
    \textbf{TODO: Discuss among all of them, critical thinking..}

    \textbf{TODO: If enough material compare the architecture relevant metrics of the approaches}