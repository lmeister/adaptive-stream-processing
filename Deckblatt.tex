% Erzeugt das Deckblatt
%   Bei zu langem Arbeitstitel m�ssen die vertikalen Abst�nde (\vspace)
%   angepasst werden, damit das Deckblatt weiterhin auf eine Seite passt.
\begin{titlepage}
\newgeometry{top=2cm,bottom=2cm,left=2cm,right=2cm}
\begin{figure}
    \includegraphics[scale=0.22]{Bilder/SSE_logo_2000_resized.pdf}
		\hfill
		\includegraphics[scale=0.25]{Bilder/St_Uni-Logo-9-2003-eps-converted-to.pdf}
\end{figure}
\begin{center}
    \vspace*{0cm}
    \huge{\textbf{\art~im Studiengang \studiengang}}
\end{center}
\begin{center}
    \vspace{1cm}
    \Huge{\textbf{\titel}}
    \vspace{1cm}
\end{center}
\ifthenelse{\boolean{final}}{}{
    \begin{center}
        \version ~vom \today\\
        (Vor Abgabe entfernen)
    \end{center}
}
\begin{center}
    \vspace{1cm}
    \huge{\textbf{\autor\\}}
    \vspace{1cm}
    \LARGE{\matnr\\}
    \vspace{0,5cm}
    \LARGE{\email}
\end{center}
\vspace{1cm}
\begin{center}
    \centering
    \LARGE{\textbf{Betreuer:}} \\
    \LARGE{\erstgutachter} \\
    \LARGE{\zweitgutachter} ~\\
\end{center}
\vspace{0.5cm}
\begin{flushright}
\end{flushright}
\begin{center}
    \small{\arbeitsgruppe \ \textbullet \ \institut \\ Universit�t Hildesheim
    \textbullet \ Universit�tsplatz 1 \textbullet \ D-31134 Hildesheim}
\end{center}
\end{titlepage}
\restoregeometry