% Formatvorlage f�r studentische Arbeiten der Arbeitsgruppe Software Systems
% Engineering am Institut f�r Informatik der Universit�t Hildesheim
%   erstellt von Christopher Voges, 28.04.2016
%   �berarbeitet von Sascha El-Sharkawy, 06.08.2019
%   https://sse.uni-hildesheim.de/studium-lehre/richtlinien-fuer-ausarbeitungen/vorlagen/

% Dokumentenkopf 
\documentclass[
    11pt, % Schriftgr��e
    DIV=10,
    ngerman, % f�r Umlaute, Silbentrennung etc.
    a4paper, % Papierformat
    twoside, % weiseitiges Dokument
    titlepage, % es wird eine Titelseite verwendet
    parskip=half, % Abstand zwischen Abs�tzen (halbe Zeile)
    headings=normal, % Gr��e der �berschriften verkleinern
    listof=totoc, % Verzeichnisse im Inhaltsverzeichnis auff�hren
    bibliography=totoc, % Literaturverzeichnis im Inhaltsverzeichnis auff�hren
    index=totoc, % Index im Inhaltsverzeichnis auff�hren
    captions=tableheading, % Beschriftung von Tabellen unterhalb ausgeben
		numbers=noenddot,
    final, % Status des Dokuments (final/draft)
]{scrreprt}

% Einbinden der Packages
% Entfernt Probleme durch Verwendung von Koma-Script Klassen (scrreport)
\usepackage{scrhack}
% Erlaubt Konfiguration von Header & Footer
\usepackage[automark,headsepline,footsepline,plainheadsepline,plainfootsepline]{scrlayer-scrpage}

% Anpassung an Sprache (deutsch)
% \usepackage[ngerman]{babel}

% Umlaute
\usepackage[latin1]{inputenc}
\usepackage[T1]{fontenc}
\usepackage{textcomp} % Euro-Zeichen etc.

% Schrift
\usepackage{lmodern}
\usepackage{relsize}

% Einbinden von JPG-Grafiken erm�glichen
\usepackage[dvips,final]{graphicx}

% Erm�glichen mathematischer Symbole
\usepackage{amsmath,amsfonts}

% F�r die Definition der Zeilenabst�nde, Seitenr�nder etc.
\usepackage{setspace}
\usepackage{geometry}

% URL-Unterst�tzung
\usepackage{url}

% Abk�rzungsverzeichnis 
% Alles weitere hierzu in: "Inhalt\Abkuerzungen.tex".
\usepackage[intoc]{nomencl}
\let\abbrev\nomenclature
\renewcommand{\nomname}{List of Abbreviations}
\setlength{\nomlabelwidth}{.15\textwidth}

% PDF-Optionen -----------------------------------------------------------------
\usepackage[
    bookmarks, % Es werden Bookmarks verwendet
    bookmarksopen=true, % Farbe von Bookmarks
    colorlinks=true, % Farbe von Verkn�pfungen
    linkcolor=black, % einfache interne Verkn�pfungen
    anchorcolor=black, % Ankertext
    citecolor=black, % Verweise auf Literaturverzeichniseintr�ge im Text
    filecolor=black, % Verkn�pfungen, die lokale Dateien �ffnen
    menucolor=black, % Acrobat-Men�punkte
    urlcolor=black, % Farbe der URLs
    plainpages=false, % zur korrekten Erstellung der Bookmarks
    pdfpagelabels, % zur korrekten Erstellung der Bookmarks
    hypertexnames=false, % zur korrekten Erstellung der Bookmarks
    linktocpage, % Seitenzahlen anstatt Text im Inhaltsverzeichnis verlinken
    pdfusetitle % Erm�glicht das Setzen der Meta-Daten des erzeugten PDFs
]{hyperref}
% \renewcommand{\theHsection}{\thepart.section.\thesection}
% \hypersetup{
%     %pdftitle={\titel \untertitel},
%     pdfauthor={\autor},
%     pdfcreator={\autor}
%     %pdfsubject={\titel \untertitel},
%    % pdfkeywords={\titel \untertitel}
% }

% Wird f�r Teile der Formatierung des Deckblatts und die Verwendung von
% Aufz�hlungen ben�tigt
\usepackage{listings}
\usepackage{xcolor} 


% fortlaufendes Durchnummerieren der Fu�noten
\usepackage{chngcntr}

% bei der Definition eigener Befehle ben�tigt
\usepackage{ifthen}

% sorgt daf�r, dass Leerzeichen hinter parameterlosen Makros nicht als Makroendezeichen interpretiert werden
\usepackage{xspace}

% F�r das Erstellen eines Glossars
\usepackage[toc, automake, nonumberlist]{glossaries}

\usepackage[figure,table,lstlisting]{totalcount}

% Einbinden der Meta-Informationen
% Meta-Informationen 
% Falls Umlaute oder ein *�* vorkommen:
\usepackage[latin1]{inputenc}

% Hier k�nnen Sie Informationen zur Arbeit, sich selbst und Ihren Betreuern
% hinterlegen.
\newcommand{\titel}{Self-Adaptive Architecture in Stream Processing} % Name der Arbeit
\newcommand{\untertitel}{Untertitel} % Optional mit Untertitil
% Art der Arbeit ggf. zus�tzlich der Titel der Veranstaltung
\newcommand{\art}{Seminararbeit} 
\newcommand{\studiengang}{Angewandte Informatik (BSc)} % Ihr Studiengang
\newcommand{\autor}{Leon Meister} % Ihr Name
\newcommand{\email}{meister@uni-hildesheim.de}% Ihre aktuelle und g�ltige E-Mail-Adresse
\newcommand{\matnr}{285631} % Ihre Matrikelnummer

% Die Angaben zu uns:
\newcommand{\institut}{Institut f\"ur Informatik}
\newcommand{\arbeitsgruppe}{Arbeitsgruppe Software Systems Engineering}
\newcommand{\erstgutachter}{MSc Cui Qin, SSE}% Ihr(e) ErstgutachterIn
% \newcommand{\zweitgutachter}{<Zweitpruefer>}% Ihr(e) ZweitgutachterIn
\newcommand{\universitaet}{Universit\"at Hildesheim\ \textbullet \ Universit\"atsplatz 1 \ \textbullet \ D-31134 Hildesheim}
\newcommand{\adresse}{\arbeitsgruppe \ \textbullet \ \institut \\ \universitaet}

\newcommand{\version}{Version 1.0}% Die Version der Arbeit

% Wird 'projektarbeit' auf 'true' gesetzt, wird keine Eigenst�ndigkeitserkl�rung
% erzeugt.
\newboolean{projektarbeit}
\setboolean{projektarbeit}{false}

% Wird 'final' auf 'true' gesetzt, werden folgende �nderungen vorgenommen:
% -Entfernen von Datum in der Kopf- und Versionsnummer in der Fu�zeile
% -Entfernen von Datum und Versionsnummer vom Deckblatt
% -Es werden Leerseiten f�r den doppelseitigen Druck eingef�gt
\newboolean{final}
\setboolean{final}{false}





% Erstellung der Verzeichnisse und Glossars aktivieren
\makeindex
\makenomenclature
\makeglossaries 
\glstoctrue 


% Kopf- und Fu�zeilen, Seitenr�nder etc. anpassen
\input{Seitenstil}

% Eigene Definitionen f�r Silbentrennung laden
\include{Silbentrennung}

% Eigene LaTeX-Befehle laden
\include{Befehle}

% Hier beginnt das eigentliche Dokument
\begin{document}
\title{\titel}
\author{\autor}

% Setzt wie tief Abschnitte numeriert und ins Inhaltsverzeichnis
% aufgenommen werden sollen, hier: bis inkl. SubSubSection (z.B. 1.1.1.1).
\setcounter{secnumdepth}{3}
\setcounter{tocdepth}{3}

% Deckblatt einbinden
% Erzeugt das Deckblatt
%   Bei zu langem Arbeitstitel m�ssen die vertikalen Abst�nde (\vspace)
%   angepasst werden, damit das Deckblatt weiterhin auf eine Seite passt.
\begin{titlepage}
\newgeometry{top=2cm,bottom=2cm,left=2cm,right=2cm}
\begin{figure}
    \includegraphics[scale=0.22]{Bilder/SSE_logo_2000_resized.pdf}
		\hfill
		\includegraphics[scale=0.25]{Bilder/St_Uni-Logo-9-2003-eps-converted-to.pdf}
\end{figure}
\begin{center}
    \vspace*{0cm}
    \huge{\textbf{\art~im Studiengang \studiengang}}
\end{center}
\begin{center}
    \vspace{1cm}
    \Huge{\textbf{\titel}}
    \vspace{1cm}
\end{center}
\ifthenelse{\boolean{final}}{}{
    \begin{center}
        \version ~vom \today\\
        (Vor Abgabe entfernen)
    \end{center}
}
\begin{center}
    \vspace{1cm}
    \huge{\textbf{\autor\\}}
    \vspace{1cm}
    \LARGE{\matnr\\}
    \vspace{0,5cm}
    \LARGE{\email}
\end{center}
\vspace{1cm}
\begin{center}
    \centering
    \LARGE{\textbf{Betreuer:}} \\
    \LARGE{\erstgutachter} \\
    \LARGE{\zweitgutachter} ~\\
\end{center}
\vspace{0.5cm}
\begin{flushright}
\end{flushright}
\begin{center}
    \small{\arbeitsgruppe \ \textbullet \ \institut \\ Universit�t Hildesheim
    \textbullet \ Universit�tsplatz 1 \textbullet \ D-31134 Hildesheim}
\end{center}
\end{titlepage}
\restoregeometry

% Setzen des Papierformats f�r den Rest des Dokuments
%\newgeometry{left=3.5cm, right=2.5cm, top=2.9cm, bottom=2.9cm}

% Bei finaler Fassung: Leere Seite als R�ckseite des Deckblatts 
\ifthenelse{\boolean{final}}{\cleardoublepage}{} 

% Selbstst�ndigkeitserkl�rung einbinden
\ifthenelse{\boolean{projektarbeit}}{}{\input{Erklaerung}}

% Bei finaler Fassung: Leere Seite als R�ckseite der Erkl�rung
\ifthenelse{\boolean{projektarbeit}}{}{\ifthenelse{\boolean{final}}{\cleardoublepage}{}}

% Abstract einbinden
% Vor dem Hauptteil werden die Seiten in gro�en r�mischen Ziffern nummeriert.
\pagenumbering{roman}

% \section*{Kurzfassung}
% \label{sec:Kurzfassung}
% Eine kurze Zusammenfassung der Arbeit, die Interesse beim Leser wecken soll. 


\section*{Abstract}
\label{sec:Abstract}
Gerne zus�tzlich oder alternativ in Englisch.


% Bei finaler Fassung: Leere Seite als R�ckseite des Abstracts
\ifthenelse{\boolean{final}}{\cleardoublepage}{}

% Verzeichnisse drucken
\tableofcontents% Inhaltsverzeichnis

% Wenn Abbildungs- und Tabellenverzeichnis auf die selbe Seite sollen: 
\iftotalfigures\listoffigures\fi
\begingroup 
\let\clearpage\relax
\vspace{1cm} 
\iftotaltables\listoftables\fi
\endgroup
 
% Wenn Abbildungs- und Tabellenverzeichnis je auf einer eigenen Seite beginnen
% sollen: 
% \listoffigures % Abbildungsverzeichnis
% \listoftables % Tabellenverzeichnis

% Erstellen des Liste der Listings
\renewcommand{\lstlistlistingname}{Quellcode-Verzeichnis}
\iftotallstlistings\lstlistoflistings\fi

% Abk�rzungsverzeichnis einbinden
\nomenclature{SPS}{Stream Processing System}
\nomenclature{DPS}{Data Stream Processing}
\nomenclature{EDF}{Elastic and Distributed DSP Framework}
\nomenclature{IDC}{International Data Corporation}

% F�r korrekte �berschrift in der Kopfzeile
\clearpage\markboth{\nomname}{\nomname}

% Drucken des Abk�rzungsverzeichnises
\printnomenclature
\label{cha:Abkuerzungsverzeichnis}
 
% Arabische Seitenzahlen im Hauptteil 
\ifthenelse{\boolean{final}}{\cleardoublepage}{\clearpage}
\pagenumbering{arabic}

% Die Inhaltskapitel aus "Inhalt.tex" einbinden
\begin{spacing}{\zeilenabstandHauptteil}
% Hier k�nnen die einzelnen Kapitel inkludiert werden. 
% Die Dateien m�ssen auf .tex enden. Diese Endung muss
% beim Inkludieren aber weggelassen werden.
% Info: \include und \input unterscheiden sich im wesentlichen darin, dass bei
% \include immer eine neue Seite angefangen wird.

\chapter{Introduction}
\label{cha:Introduction} % Ein Label ist optional, ermoeglicht aber die Referenzierung
\textbf{\color{red}THIS IS NOT FINAL}

In this chapter we will explore the motivation and goals of this seminar thesis, as well its structure.
In order to do this, we will explain the motivation behind the topic as well as the research questions that we aim to answer in section \ref{sec:motivation-goals} \nameref{sec:motivation-goals}.
Afterwards we will then explain the thesis's structure and contents in section \ref{sec:structure} \nameref{sec:structure}.

\section{Motivation and Goals}
\label{sec:motivation-goals}
This chapter delves into the motivation behind this thesis and furthermore defines research questions, which we aim to answer in the subsequent chapters.

% Introduction: lots of data -> cant save it all -> need a way to deal with it
% Examples of when it is applied
% How does it work, are there alternatives? yes -> batch processing (Maybe briefly explain)
The advancements in technology of the past decades has lead to enormous data creation. Technology has become ubiquitous, 
with the evolution of cell phones to smartphones, the digitilization of industrial processes, Industry 4.0
and the increasing amount of ``smart`` devices, causing creation of information to grow exponentially.
It is estimated that the \gls{datasphere} will reach the size of 175 zettabytes by 2025, as shown in figure \ref{fig:growth_datasphere}.
% Graph estimating the growth of the global datasphere (by IDC)
\begin{figure}[ht]
\centering
\includegraphics[width=1.0\textwidth]{Bilder/size_global_datasphere.png}
\caption{The Growth of the Global Datasphere \cite[p.6]{idc-seagate-data}}
\label{fig:growth_datasphere}
\end{figure}`

% Maybe a source here?
Data has become an important factor in decision making and optimization in virtually every industry, especially in finances. \textbf{TODO: Wieso?}
The financial market is dominated by data driven decisions, with emphasis on data processing in a (near) real-time fashion.
However, real-time data is becoming of importance in multiple sectors; the International Data Corporation estimates that real-time data will be 
responsible for a share of 30 percent of the total global datasphere by 2025, as shown in figure \ref{fig:growth_realtime_data}.
% Graph showing the growing share of real-time data as part of global datasphere
\begin{figure}[h]
\centering
\includegraphics[width=1.0\textwidth]{Bilder/realtime_data.png}
\caption{The growth of real-time data as part of the Global Datasphere \cite[p.13]{idc-seagate-data}}
\label{fig:growth_realtime_data}
\end{figure}

A global study led by IBM in 2012 has shown that 71 percent of the firms in the financial market use information (including big data)
in order to achieve an advantage over their competitors, compared to 36 percent, which IBM has found in an earlier study conducted in 2010. \cite[p.1]{ibm-financial}

As it is no longer feasible to save all the data before then analyzing it (in batches), due to computational cost and lack of storage capacity, 
a new approach was designed in order to handle data in a (near) real-time fashion, Stream Processing Systems (SPS).

However the rate in which data is being streamed fluctuates, so one comes to the logical conclusion that SPSs should adapt to the velocity and volume of the stream.
Besides the stream, the computing environment might also fluctuate, e.g. a resource failing in a distributed computing environment, 
and as such the system has to adapt. There are many different techniques of adaptation such a system might utilize, 
which can impact both the stream as well as the SPS. We will elaborate on this in \ref{sub:sps} \nameref{sub:sps}.


\textbf{TODO: Why self adaptivity is needed? Research different kinds of adaptation (e.g. load shedding, elastric allocation of resources), look at cui's paper, add concrete examples
Structure:
Why stream processing? -> Why adaptive?}

RESEARCH QUESTIONS:

1. What is Stream Processing?

2. What are possible techniques for adapatation in stream processing?

3. Why is self-adaptivity needed in Stream processing?

4. What are current approaches in self-adaptive stream processing?


\section{Structure}
\label{sec:structure}
Explain Structure and contents of chapters here

\chapter{Fundamental Concepts}
\label{cha:fundamentals}
\textbf{TODO: I -> We, no need to always say will, sometimes use passive to switch it up}
In this chapter we lay out the fundamental concepts, which are necessary in order to understand the subsequent chapters.
First we explain Stream Processing in \ref{sec:stream-processing}, afterwards the concept of the self-adaptive systems will be presented and explained in section \ref{sec:self-adaptive} 
followed by a brief explanation of the MAPE-K loop in \ref{sub:mape} to conclude this chapter.

    \section{Stream Processing}
    \label{sec:stream-processing}
    In this section we will split the concept of stream processing into three further components.
    In \ref{sub:sps} we will then define stream processing systems, explain how they work and give examplary fields of application.
    Afterwards in \ref{sub:dsms} we will move onto the topic of data stream management systems and finally in \ref{sub:requirements} we talk about
    some requirements that SPSs should meet.
    
        \subsection{Data Stream Management Systems}
        \label{sub:dsms}
        \ifbool{final}{}{\textbf{\color{green}THIS IS FINAL}
        
        }
        % - Streams not saved to disk (Saving to disk and retrieving has high latency), instead kept in memory (low latency)
        % - Some data might still be archived, however it does not impact performance, as it is done parallel to the streaming processing
        % - Sometimes access to archived data in DBMS needed, however this can be cached or efficiently accessed

        \begin{figure}[h]
            \label{fig:dbms_dsms}
            \centering
            \includegraphics[width=1.0\textwidth]{Bilder/dbms_dsms.png}
            \caption{
                   left: architecture of a DBMS, right: architecture of a DSMS
                   }
        \end{figure}

        A data stream management system is a tool designed to manage continuous data streams.
        DSMSs are comparable to DBMS, however there are differences one must take note of.

        Most notably the data being fed into a DSMS varies greatly from the data being inserted into a DBMS; while a DBMS receives a finite predetermined amount 
        of data, a DSMS could theoretically receive an infinite continuous stream.

        Secondly, DSMSs operate at a much greater speed than traditional DBMSs, because the data is processed immediately, while in the DBMS approach
        data is first saved to a disk and then queried before further processing can occur, leading to two I/O actions. This behavior, coupled with an incoming 
        high velocity stream, might lead to lag or even failure to yield accurate results\cite{StreamBookQuality}. 
        However, a DSMS might want to archive certain elements from the stream. There are two options that are considered;
        Either data is being cached in memory, yielding much faster query results, or it is archived in additional databases parallel to the streaming process.
        Thus the DSMS architecture is superior to the DBMS architecture when measuring latency.
        When data is stored, often only synopses are stored, in an attempt to summarize the data (and utilize that knowledge for the queries.)
        The difference in these architectures is also shown in figure \ref{fig:dbms_dsms}\cite{StreamBookQuality}.

        Thirdly, similar to a DBMS a DSMS can also be queried using a streaming query language, however, queries installed on a DSMS are continuous 
        and will be executed as long as it is uninstalled and there are restrictions on which operations can be executed.
        Since the system can not assume a finite amount of data, it will process each data item as it arrives, so operators that require the entirety of a data 
        set, such as the \textit{join, aggregation} operators, prove problematic when it comes to streams, as they can only return a result once the stream has ended\cite[p.12]{StreamBookQuality}.
        These queries are refered to as \textit{blocking} queries. In order to use them, one must convert them into non-blocking queries.
        One solution for this has been introduced in the form of windows, which are explained in \ref{sub:sps} \nameref{sub:sps}.

        There are a few more functional differences, which Panigati, Schreiber and Zaniolo highlight, as shown in table 2.1.

        \begin{table}[h]
            \centering
            \label{tab:dbms-dsms}
            \begin{tabular}{|c|c|c|} \hline
                \textbf{Feature} & \textbf{DBMS} & \textbf{DSMS} \\ \hline
                Model & Persistent data & Transient Data \\ \hline
                Table & Set or bag of tuples & Infinite sequence of tuples \\ \hline
                Updates & All & Append only \\ \hline
                Queries & Transient & Persistent \\ \hline
                Query answers & Exact & Often approximate \\ \hline
                Query evaluation & Blocking and non-blocking & Non-blocking \\ \hline
                Operators & Fixed & Adaptive \\ \hline
                Data processing & Synchronous & Asynchronous \\ \hline
                Concurrency overhead  & High & Low \\ \hline
            \end{tabular}
            \captionbelow{Functional comparison of DBMS and DSMS\cite{Panigati2015}} 
        \end{table}

        \subsection{Stream Processing Systems}
        \label{sub:sps}
        \ifbool{final}{}{\textbf{\color{red}THIS IS NOT FINAL}
        
        }
        \textbf{TODO: Explain big data -> batch(latency low prio) and stream(latency high prio) }
        In this subsection we will explain what SPSs are, what their input looks like, what SPSs are able to do and give an example.
        In the end of this subsection we will also go over the challenges that SPSs face.

        % 1. Mehrere Generatoren generieren datenstroeme
        % -> Wie sieht ein Stream aus (infinite amount of tuples? see literature)
        % -> Was kann ein "Generator" sein? GPS Sensoren, EKG, EEG, .. (give few examples)
        Stream processing systems are fed continuous data streams, generated by data providers, for example GPS-sensors or EEG- and EKG-machines.
        Per definition of Panigati, Schreiber and Zaniolo\cite{Panigati2015} a data stream $S$ consists of a infinite, countable series of four-tuples $s := <\nu, t^{app }, t^{sys}, b_{id}>, s \in S$, where:

        \textbf{TODO: Replace this with a more general definition, e.g. from fundamentals of stream processing book}
        \begin{itemize}
        \item $\nu \in \mathfrak{R}$ is a relational tuple

        \item $t^{app} \in T$, with $T$ being the time domain, is a partially ordered application time
        
        \item $t^{sys} \in T$,  with $T$ being the time domain, is a totally ordered system time
        
        \item $b_{id} \in B$ is a batch id; batch $B$ is a finite subset of $S$ where all $b \in S$ have an identical $t^{app}$
        \end{itemize}

        % 2. Diese werden im SPS verarbeitet
        % -> Beispiel filtering        
        Each element \textit{s} then is processed by one or more operators, each doing their own independent operation, e.g. filtering.
        and then put out into the operator's outgoing stream.

        % -> Windows erklären (Tumbling, sliding) (grafiken zum Vergleich zeigen)
        \textbf{TODO: Implement some structure, not own sections but little headers maybe or new paragraphs or something}
        -> explain windows

        -> exanpe aggregation Tumbling

        -> example aggregation sliding

        -> elaborate Difference

        % -> Replication of operators zeigen
        In order to increase efficiency, an SPS can, if (computational) resources are available, create replicas of operators to introduce parallelity, as shown in figure \ref{fig:sps_parallel_normal} \textbf{Ref wrong?}. 
        Conversely, if there is little input, it may also reduce the amount of replicas in order to save or free up additional resources.

        
        % 3. SPS kann gequeried werden
        % -> Eventuell Beispiel geben
        
        % 4. SPS gibt Stream aus, welcher auf verschiedene Arten weiter verwaltet wird
        % -> Resultate können bspw. gespeichert werden, siehe graph
        % -> Outputstream kann auch weiter verarbeitet werden von anderen Systemen

        % 5. Challenges
        % -> e.g. Correctness 

        \begin{figure}[h]
            \label{fig:sps_parallel_normal}
            \centering
            \includegraphics[width=1.0\textwidth]{Bilder/sps_parallel_normal.png}
            \caption{
                    Left: An example for an SPS displayed as a directed acyclic graph. 
                    Right: Same SPS with introduced parallelity in one operator, marked gray for visibility. 
                    Circles are input/outputs, squares are operators, arrows are streams.
                    TODO: EXPLAIN graph and sink, generator, processor!!
                    }
        \end{figure}

        \begin{figure}
        \label{fig:stream-processing-system}
        \centering
        \includegraphics[width=1.0\textwidth]{Bilder/stream-processing-system.png}
        \caption{
                Overview of a basic Stream Processing System
                }
        \end{figure}  

        % Subsection Requirements for Stream Processing Systems
        % Should quickly go over the requirements and explain why they matter to us
        \subsection{Requirements for Stream Processing Systems}
        \label{sub:requirements}
        \ifbool{final}{}{\textbf{\color{red}THIS IS NOT FINAL}
        
        }

        Notes: What are the requirements, why do they matter to us (Elaborate on this)

        Due to the nature of the fields in which SPS are used, there are important requirements that SPS should meet in order to be viable, 
        which Stonebraker et al. point out in \cite{Stonebraker:2005:RRS:1107499.1107504}, of which the ones most important to us can be summarized as the following:
        
        % Enumeration of requirements for SPS + explanations why they matter to us
        \begin{enumerate}
        \label{enum:requirements}
            \item \textbf{Keep the Data Moving:} 
                In order to minimize latency, data must not be stored, as these are costly operations.
            \item \textbf{Handle Stream Imperfections:} 
                Expecting only perfect data is utopian, so one must prepare the system with built-in mechanisms for data that might be missing or out-of-order.
            \item \textbf{Integrate Stored and Streaming Data:} 
                For an SPS to be able to perform comparisons between "predecessor" data and current data, operators must keep an efficiently manageable state.
            \item \textbf{Guarantee Data Safety and Availability:} 
                Recovering from a failure is detrimental for real-time data processing, so a system must be in place to guarantee the highest availability possible.
            \item \textbf{Process and Respond Instantaneously:} 
                Systems must be highly optimized in order to provide (near) real-time responses.
            \item \textbf{Partition and Scale Applications Automatically:} 
                Systems must be able to be split across multiple machines and threads.
                The system must also be able to automatically scale and distribute the load across the machines.

        \end{enumerate}

    \section{Self-Adaptive Systems}
    \label{sec:self-adaptive}
    \ifbool{final}{}{\textbf{\color{green}THIS IS FINAL?}
    
    }{}
    
    % \textbf{TODO: Needs more; e.g why self-adaptive(already mentioned in intro), Software engineering perspective on self adaptivity, challenges}
    % Definition of self-adaptive systems
    % Architecture often based on mape in different patterns
    % applied in xx industries
    Cheng et al. define self-adaptive systems as
    \begin{quotation}
        ``[...] systems that are able to adjust their behaviour in response to their perception of the environment and the
        system itself [...]``\cite[p.1]{Cheng:2009:SES:1573856.1573858}.
    \end{quotation}
    
    Self-adaptive systems are oftentimes based on the \nameref{sub:mape} [p.\pageref{sub:mape}] pattern.
    Adaptive Systems have a wide variety of possible application areas: adaptable user interfaces, autonomic computing, multi-agent systems \cite{Cheng:2009:SES:1573856.1573858}, 
    biologically inspired computing, robotics \cite{10.1007/978-3-319-59480-4_44}, streaming applications and a lot more.

    An examplary application would be a scenario, in which population and food capacities are given and evolving over time, due to births, deaths, changes in demographics 
    and changes in weather and harvest respectively. A system would have to adapt to these changes in its environment in order to ration the food properly.

    
    \subsection{MAPE-K Loop}
    \label{sub:mape}
    \ifbool{final}{}{\textbf{\color{green}THIS IS FINAL (Maybe add something underneath enum if not sufficient)}
    
    }{}
    % Explain the MAPE-K Loop as it is a valuable basis/reference architecture for many different approaches in adaptive systems
    % Rough explanation of what it is, where it is used
    % explain different "stages" (m, a, p, e)
    % Explain -K extension
    The MAPE-K Loop was introduced by IBM \cite{Kephart:2003:VAC:642194.642200} and refers to a proposed solution for self-adaptive or autonomic systems.
    This model has since become the basis or reference architectural pattern for many self adaptive systems, which I will show in the third chapter.
    The acronym MAPE-K refers to the components that make up the model:
    \textbf{TODO: Add a diagram, make this a subsection of self-adaptive systems}
     \begin{figure}[hbt]
        \label{fig:mape}
        \centering
        \includegraphics[width=0.45\textwidth]{Bilder/mape.png}
        \caption{
                Overview of the MAPE-K-Loop\cite{Kephart:2003:VAC:642194.642200}
                }
    \end{figure}  
    \begin{enumerate}
    \label{enum:mape}
        \item \textbf{M}onitor: 
            The \textit{Monitor} component gathers data about the system and its environment, aggregates and filters it.
        \item \textbf{A}nalyze: 
            The \textit{Analyze} component analyzes the previously gathered data and determines whether or not an adaptation should be performed.
            The decision is made based on performance or cost gain and should include the adaptation cost as well.
            This component's analysis is influenced by the \textit{Knowledge} base.
        \item \textbf{P}lan: 
            If the choice to adapt the system has been made, the \textit{Plan} component then decides how to reconfigure the system.
            Once the decision has been made, the information is then forwarded to the \textit{Execute} component.
        \item \textbf{E}xecute: 
            Given the \textit{Plan} component's decision, the \textit{Execute} component then executes said plan and the loop 
            returns to the initial monitoring state.
        \item \textbf{K}nowledge: 
            Represents the knowledge base, which is shared between the other components.
            This base is created by the \textit{Monitor} component and contains information in the form of metrics, policies, symptoms and logs.
    \end{enumerate}
   


\chapter{Approaches for Self-Adaptive Architectures in Stream Processing}
\label{cha:approaches}
\textbf{TODO: How does it work? FOCUS ON ARCHITECTURE IN THIS CHAPTER!!}
Explain that this chapter showcases a few select strategies, which are then elaborated on further in the subchapters
Question: Even more approaches? e.g. Master-Slave pattern or Coordinated Control pattern (Both MAPE based)?
\textbf{Add and explain a few more MAPE Based architectures}

    \section{Dhalion}
    \label{sec:dhalion}
    Quick Introduction to Dhalion, this chapter will deal with the Dhalion paper.
    NOTE: Explain what Dhalion is, where its used

        \subsection{An Outline of Heron}
        \label{sub:heron-outline}
        Small outline of Heron, as Dhalion is built on top of Twitter's Heron.

        \subsection{Dhalion's Architecture}
        \label{sec:dhalion-architecture}
        \textbf{TODO: Diagram, explain it here}
        Explanation of Dhalion's Architecture \textbf{KERNPUNKT DER SECTION DHALION}

        \subsection{Discussion of Dhalion}
        \label{sec:dhalion-discussion}
        Discuss the approach and compare it to the reference architecture (Mape?)
        \textbf{TODO: Maybe discuss how they evaluate, look at metrics relevant to architecture}

    \section{Decentralized Self-Adaptation}
    \label{sec:hierarchical}
    \textbf{TODO: Incorporate same structure as above}
    NOTE: In this section I will explain the hierarchical control architecture as decribed by Cardellini..

        \subsection{Elastic and Distributed DSP Framework}
        \label{sec:edf}
        \textbf{TODO: Possibly change title of this, add diagram, explain thoroughly (decentralized etc.), extract their design patterns}
        Explanation of the EDF, their architecture for elastic DSP apps \textbf{KERNPUNKT DER SECTION}

        \subsection{Discussion of EDF}
        \label{sec:discussion-edf}

    \section{Some Other Architecture}
    \label{sec:soa}
    NOTE: In this chapter I will explain another architecture/approach to self-adaption, yet to be researched


    \section{Some Other Architecture2}
    \label{sec:soa2}
    NOTE: In this chapter I will explain another architecture/approach to self-adaption, yet to be researched


    \section{Title??}
    \textbf{TODO: Discuss among all of them, critical thinking..}

    \textbf{TODO: If enough material compare the architecture relevant metrics of the approaches}
\chapter{Summary And Conclusion}
\label{cha:Summary And Conclusion} % Ein Label ist optional, erm�glicht aber die Referenzierung
\section{Summary}
Summarize the paper
\section{Conclusion}
Conclude the paper


\end{spacing}

% Die Inhalte des Anhangs werden analog zu den Kapiteln eingebunden.
% Wenn kein Anhang vorhanden ist, m�ssen die n�chten 7 Zeilen auskommentiert
% werden (bis '\end{spaching}').
\begin{spacing}{\zeilenabstandAnhang}
\appendix
    \chapter{Anhang}
    \label{sec:Anhang}
    % Hier k�nnen die einzelnen Anh�nge inkludiert werden. 
% Die Dateien m�ssen auf .tex enden. Diese Endung muss
% beim inkludieren aber weggelassen werden.
% Info: \include und \input unterscheiden sich im wesentlichen darin, dass bei
% \include immer eine neue Seite angefangen wird.
\input{Inhalt/Beispiele} % Vor finaler Abgabe entfernen!

\end{spacing}

% Glossar einbinden
% Ein Beispiel f�r einen Glossareintrag
% Nur Eintr�ge, die mindestens ein Mal referenziert wurden 
% (z.B.: \gls{computer}), tauchen im Glossar am Ende der Arbeit auf.
\newglossaryentry{datasphere} {
   name=Global Datasphere,
   description={The Global DataSphere quantifies and analyzes the amount of data created, captured, and replicated in any given year across the world [SOURCE https://www.idc.com/getdoc.jsp?containerId=IDC_P38353]}
}
\newglossaryentry{state} {
   name=State,
   description={Some explanation for state TODO}
}
\newglossaryentry{tweet} {
   name=tweet,
   description={Some explanation for tweet TODO}
}

\label{sec:Glossar}

% Mit diesem Befehl werden nach nicht referenierte Glossareintr�ge
% ausgegeben (Nur zum Testen einkommentieren!)
% \glsaddall

% Glossar drucken
\printglossaries
% \addcontentsline{toc}{chapter}{Glossar}

% \ifthenelse{\boolean{final}}{\cleardoublepage}{\clearpage}
\clearpage
% \addcontentsline{toc}{chapter}{Bibliography}


% Literaturverzeichnis
%   Quelldatei ist: "Bibliographie.bib".
\bibliography{Bibliographie} % Aufruf: bibtex

% Verschiedene Presets f�r den Stil der Zitate und des Literaturverzeichnisses
% \bibliographystyle{plaindin} % Nur Zahlen, deutsch
% \bibliographystyle{plain} % Nur Zahlen, englisch
\bibliographystyle{alphadin} % Alphanumerische K�rzel, deutsch
% \bibliographystyle{alpha} % Alphanumerische K�rzel, englisch

\end{document}